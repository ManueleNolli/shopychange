\problemchapter{Abstract}

L'adozione di tecnologie \textit{Blockchain} sta crescendo rapidamente, un settore di interesse è lo scambio di beni. La causa di questa crescita così rapida è la fiducia, spessa spropositata, che i marketplace centralizzati richiedono agli utilizzatori. Sempre più aziende apprezzano la sicurezza e la trasparenza fornite dalla tecnologia basata su registro distribuiti.

Il presente documento analizza lo stato dell'arte dei marketplace, sia centralizzati che decentralizzati, e la tecnologia \textit{Blockchain}, proponendo una soluzione basata su \textit{Ethereum} per la creazione, la vendita e la gestione di beni digitali. 
Verranno prese in considerazione, anche in modo dettagliato, tematiche attuali e particolarità della tecnologia decentralizzata, tra queste il concetto di \textit{revenue share} in modo automatizzato, ovvero la possibilità di garantire il pagamento automatico al creatore di un bene ad ogni rivendita. 
Inoltre, verrà descritto in dettaglio il funzionamento del recupero dei dati presenti nella \textit{Blockchain} e le diverse alternative per la creazione di beni digitali.
Tutto ciò con l'intento di proporre un \textit{marketplace decentralizzato}, ovvero una \textit{Decentralized App}, nel contesto dell'evoluzione verso il \textit{Web 3.0}. Il progetto mira a risolvere il problema di fiducia e di trasparenza nei sistemi centralizzati, sfruttando le potenzialità di una nuova era di interazione digitale.

I risultati ottenuti dimostrano che la soluzione proposta è in grado di risolvere il problema fin da subito e che la tecnologia \textit{Blockchain} è un'ottima alternativa ai sistemi attualmente più popolari, ovvero quelli centralizzati.