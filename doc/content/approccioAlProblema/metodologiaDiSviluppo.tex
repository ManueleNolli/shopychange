\section{Metolodogia di sviluppo}
Il progetto è stato sviluppato utilizzando la metodologia \textit{Agile}, in particolare è stato utilizzato il framework \textit{Scrum}, in modo da poter avere un'idea chiara dei requisiti e delle funzionalità che il sistema deve avere, così da poterle implementare in modo incrementale e iterativo. 

Gli \textit{Sprint} sono stati di due settimane, così da avere un costante feedback da parte del relatore.

\subsection{Git}
Lo strumento di versionamento utilizzato è \textit{Git} o più precisamente \textit{GitLab}. Questo strumento offre anche diverse funzionalità di supporto all'utilizzo di metodologie Agile. Infatti, i \textit{milestone} sono stati utilizzati per identificare gli \textit{Sprint} e le \textit{issue} per identificare i singoli task da svolgere. Queste ultime sono state divise in \textit{User Story} con lo scopo di essere connesse ai requisiti funzionali e non.

\subsection{CI/CD}

Grazie a \textit{Gitlab} è stato possibile utilizzare il \textit{Continuous Integration} e il \textit{Continuous Delivery} per automatizzare il processo di testing. Attraverso un file di configurazione chiamato \textit{.gitlab-ci.yml} è stato possibile calcolare la \textit{coverage} del codice e verificare che tutti i test avessero esito positivo. 