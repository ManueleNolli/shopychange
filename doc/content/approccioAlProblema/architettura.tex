\section{Architettura}
\label{sec:architettura}

Di seguito viene presentata l'architettura del sistema prodotto. La figura \ref{fig:architettura} mostra come le varie componenti interagiscono tra loro.

\begin{figure}[H]
    \centering
    \includegraphics[width=1.0\textwidth]{images/infrastruttura}
    \caption{Architettura del sistema}
    \label{fig:architettura}
\end{figure}

Come si può notare il sistema è composto da diversi componenti chiave. 

Il \textit{frontend} è stato sviluppato utilizzando il framework \textit{React} e il linguaggio di programmazione \textit{Typescript}. Esso si occupa di gestire l'interazione con l'utente, in particolare si occupa di richiedere i dati al backend e di mostrare all'utente le varie informazioni, nonché di inviare le richieste di interazione alla blockchain.

Come è osservabile dalla figura \ref{fig:architettura}, il frontend comunica con il backend tramite delle API di tipo GraphQL. Queste API sono state sviluppate utilizzando il framework \textit{Django} e il linguaggio di programmazione \textit{Python}. Il backend si occupa di gestire le richieste provenienti dal frontend, ovvero di recuperare i dati presenti nella blockchain e nel database, nonché di elaborarli e restituirli al frontend.

Tramite l'utilizzo del framework \textit{Hardhat} è stato possibile sviluppare gli smart contracts, i quali sono stati scritti utilizzando il linguaggio 
\textit{Solidity}. Inoltre, sono stati compilati e distribuiti sulla blockchain di test \textit{Sepolia}. Essi si occupano di gestire la logica di \textit{business}, in particolare la creazione, modifica e vendita di un asset, nonché la gestione del revenue share. In aggiunta, il framework permette di creare un'istanza di un nodo locale, così da poter utilizzare una blockchain locale.

Infine, il database è \textit{MongoDB}, ovvero un database di tipo \textit{NoSQL} basato su documenti. All'interno di esso vengono salvati unicamente alcuni dati di personalizzazione utente del marketplace.

In figura \ref{fig:architettura}, sono presenti due frecce tratteggiate, di seguito la loro spiegazione:
\begin{itemize}
    \item \textit{(1)}: L'esperienza utente è completamente compatibile sia con \textit{Desktop Browser} che con \textit{Mobile Browser}, tuttavia la connessione di un \textit{wallet} su \textit{Mobile Browser} presenta una funzionalità ridotta. La causa potrebbe essere dovuta da un problema di \textit{deep linking mobile}, ovvero la possibilità di aprire una comunicazione diretta tra due applicazioni, in questo caso tra il \textit{wallet} e il \textit{browser}. Questo problema è stato riscontrato utilizzando il \textit{wallet} \textit{MetaMask} su dispositivo \textit{Android}. Una possibile soluzione è quella di mettere in sicurezza l'applicativo frontend tramite l'utilizzo di certificati, in modo da poter utilizzare il protocollo \textit{https}.  
    \item \textit{(2)}: Non è attualmente presente un database di \textit{caching} dei dati presenti nella blockchain. In caso di forte utilizzo del marketplace in una blockchain pubblica e principale, come Ethereum mainnet, i tempi di risposta potrebbero essere molto alti. Perciò, sarebbe necessario implementare un database di questo tipo, un possibile candidato è \textit{Redis}.
\end{itemize}

Maggiori informazioni riguardo ad ogni componente e le motivazioni dietro le scelte effettuate sono presenti nei capitoli successivi.