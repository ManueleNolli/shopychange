\chapter{Conclusioni}

All'interno di questo capitolo verranno presentate le conclusioni del lavoro svolto, discutendo i risultati ottenuti, i problemi riscontrati e proponendo possibili sviluppi futuri.

Il progetto ha avuto come obiettivo la realizzazione di un marketplace decentralizzato per migliorare i processi di vendita, rendendoli più sicuri e trasparenti. Il lavoro svolto ha permesso di creare un luogo di scambio consono alle esigenze di creatori, venditori e acquirenti, senza la necessità di intermediari.

Come discusso nel capitolo \hyperref[sec:risultati]{\textit{Risultati}}, la tecnologia \textit{Blockchain} permette un approccio rivoluzionario allo scambio di beni, garantendo diverse migliorie rispetto ai sistemi tradizionali. Attualmente lo strumento sviluppato è adatto a diversi business, in particolare a quelli che necessitano una maggiore trasparenza e sicurezza. Alcuni esempi possono essere i mercati del gaming e dell'arte digitale. Ma con delle varianti, riguardanti il salvataggio dei metadati degli asset, il sistema potrebbe essere adatto anche per il settore di licenze su software, musica, video e molto altro. Negli ambiti appena citati il concetto di \textit{revenue share} risulterebbe fondamentale, in quanto permetterebbe di garantire un guadagno continuo ai creatori degli asset, anche dopo la vendita iniziale. Tuttavia, prevedo che la tecnologia \textit{Blockchain} migliorerà e si svilupperà ulteriormente per poter essere adottata su larga scala, con lo scopo di avere una maggior interoperabilità tra i diversi sistemi creati su di essa.

A livello professionale il progetto mi ha consentito di approfondire diverse conoscenze e di sviluppare nuove competenze. In particolare, ho avuto modo di migliorare le mie capacità di analisi, di progettazione e di sviluppo. Ho anche arricchito la mia capacità di affrontare situazioni decisionali e di effettuare ricerca scientifica. 

Il progetto ha favorito la mia crescita professionale, scaturita dal conseguimento di nuove competenze o dall'approfondimento di conoscenze ottenute durante il percorso di studi. In particolar modo grazie alle tecnologie e linguaggi utilizzati, ovvero \textit{React}, \textit{Solidity} e \textit{Python framework}.

\newpage

Sono soddisfatto del lavoro svolto e dei risultati ottenuti. Sono sicuro di aver contribuito alla realizzazione di un sistema che, un giorno, migliorerà l'approccio della società verso i beni digitali e fisici, attraverso un sistema equo e trasparente. Come disse Hal Finney: \footnote{Computer scientist e criptografo, è stato il primo destinatario di una transazione Bitcoin}

\begin{quote}
    \centering
\textit{"The computer can be used as a tool to liberate and protect people, rather than to control them"}
\end{quote}
    
\section{Problemi noti}

La soluzione proposta presenta un unico problema noto, ovvero la connessione tra un \textit{device} mobile ed un wallet. Come descritto nel capitolo \hyperref[sec:architettura]{\textit{Architettura}}, il problema è presumibilmente causato dal \textit{deep linking} tra più applicazioni. Una possibile soluzione è quella di includere un certificato all'interno dell'applicativo frontend, in modo da poter utilizzare il protocollo \textit{https}.

\section{Sviluppi futuri}

Gli sviluppi futuri potrebbero essere molteplici e toccare diversi aspetti del progetto. Di seguito sono elencate alcune funzionalità che potrebbero essere implementate:

\begin{itemize}
    \item Caching dei dati presenti nella blockchain, in modo da ridurre i tempi di risposta.
    \item Possibilità di concedere la gestione di un NFT ad un altro utente.
    \item Creazione di un sistema di asset preferiti.
    \item Barra di ricerca per il filtraggio degli asset.
    \item Creazione di una collezione di NFT con \textit{freeze}, ovvero la possibilità di non poter aggiungere asset alla collezione.
    \item Estendere in concetto di NFT ad asset che non siano unicamente immagini.
\end{itemize}


