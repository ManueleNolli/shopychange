\section{Web 3.0}
\label{sec:web3}
Il termine Web 3.0 è stato coniato dal cofondatore di Ethereum Gavin Wood nel 2014. Wood idealizzò una soluzione per uno dei principali problemi del Web tradizionale: troppa \textit{fiducia} è richiesta. La maggior parte dei siti tradizionali, come i \textit{marketplace centralizzati}, si basano sul presupposto di porre fiducia che una serie di compagnie private agisca nel pieno interesse pubblico. \cite{web3-ethereum}

Le caratteristiche che contraddistinguono Web 3.0, o più semplicemente Web3, sono:
\begin{itemize}
    \item Decentralizzazione: al posto di grandi aree di Internet controllate da entità centralizzate, la proprietà di Web3 è distribuita tra i suoi creatori e i suoi utenti.
    \item Non prevede permessi e privilegi: l'accesso a Web3 è uguale per tutti coloro che vogliono accedervi, nessuno escluso.
    \item Prevede nativamente i pagamenti: invece di appoggiarsi a banche ed elaboratori di pagamento, Web3 utilizza criptovalute per spendere e inviare denaro online.
    \item Non si basa sulla fiducia: opera usando incentivi e meccanismi economici invece di affidarsi a terze parti fidate.
\end{itemize}
Ciò che rende importante il Web3 è la possibilità di possedere risorse digitali in maniera diretta, senza bisogno di intermediari, oltre a garantire resistenza alla censura da parte di entità centralizzate.
Il Web3 necessita di una tecnologia distribuita per poter funzionare: \textit{Blockchain}. \cite{web3-ethereum}
