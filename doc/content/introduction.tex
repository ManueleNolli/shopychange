\problemchapter{Introduzione}

%motivazione e contesto
Il presente documento costituisce la documentazione del progetto di diploma svolto dallo studente Manuele Nolli e ha lo scopo di fornire una descrizione esaustiva delle attività svolte.

%problema
\vspace{5mm} %5mm vertical space
L'obiettivo di questo progetto mira alla realizzazione di un marketplace basato su tecnologia \textit{Blockchain}, con lo scopo di risolvere il problema di fiducia e di trasparenza nei sistemi centralizzati. Il testo guida il lettore partendo dall'analisi dello stato dell'arte fino alla soluzione proposta. 

% stato dell'arte
\vspace{5mm} %5mm vertical space
Il mercato digitale globale è in costante crescita e sempre più aziende considerano l'idea di adottare tecnologie \textit{Blockchain} per rendere i propri processi di vendita più sicuri e trasparenti. Per questo motivo all'interno del documento verranno analizzate a fondo le caratteristiche dei marketplace centralizati e non, includendo le caratteristiche delle tecnologie \textit{Blockchain} ma concentrandosi su \textit{Ethereum} e i vari standard \textit{NFT} (Non-Fungible Token). Verranno presentati i principali marketplace decentralizzati e le loro soluzioni per la gestione delle \textit{royalties}.

% approccio al problema
\vspace{5mm} %5mm vertical space
Il progetto include la realizzazione di una \textit{Decentralized App}, ovvero un marketplace decentralizzato basato su \textit{Ethereum} che permetta la creazione, la modifica, la vendita e l'acquisto di beni digitali. Il marketplace deve essere in grado di gestire il concetto di \textit{revenue share} per garantire il pagamento automatico ai creatori di beni digitali ad ogni rivendita. La soluzione proposta deve essere integrabile con \textit{Blockchain} private e pubbliche.


% risultati e conclusioni
\vspace{5mm} %5mm vertical space
Nella fase conclusiva del progetto verranno esposti i risultati ottenuti dalla realizzazione del marketplace e le riflessioni maturate durante lo sviluppo del progetto. L'obiettivo finale è quello di dimostrare che la soluzione proposta è in grado di risolvere il problema di fiducia e di trasparenza nei sistemi centralizzati. 



